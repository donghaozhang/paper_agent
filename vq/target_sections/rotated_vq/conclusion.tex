\section{Conclusion}

The development of the Rotated Vector Quantization framework addresses critical challenges in representation learning, particularly in maintaining reconstruction fidelity and optimizing codebook usage within the VQ-VAE architecture. Key contributions include the introduction of a novel rotation mechanism during discretization, a refined ResNet for feature encoding, and an innovative gradient propagation approach, leading to enhanced latent representations and superior reconstruction outputs. Experimental evaluations demonstrate significant improvements on CIFAR-10 and ImageNet, showcasing a reconstruction loss of 0.0098 and a codebook utilization of 96.8\%, underscoring the effectiveness of our method over traditional models. Future research should focus on further refining the rotation mechanism and exploring advanced architectural innovations, particularly in the context of integrating attention mechanisms to enhance model efficiency and representation capabilities in increasingly complex real-world datasets.