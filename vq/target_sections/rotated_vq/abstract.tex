\begin{abstract}

Recent advancements in generative modeling and representation learning have significantly enhanced the ability to generate high-quality images and compress complex datasets. However, challenges such as image reconstruction and quantization persist, largely due to existing vector quantization methods that struggle with fidelity during reconstruction and efficient codebook utilization. These limitations often result in codebook collapse and suboptimal representation quality, complicating performance validation and impeding innovation in representation learning. In response to these challenges, we introduce the Rotated Vector Quantization framework, which builds upon the Vector Quantized Variational Autoencoders (VQ-VAE) by integrating a refined ResNet architecture for feature encoding, a unique rotation mechanism for discretization, and a custom gradient propagation strategy to address non-differentiable quantization issues. Our empirical results demonstrate that this innovative approach consistently outperforms traditional VQ-VAE models, achieving significantly lower reconstruction loss, enhanced codebook utilization, and improved representational fidelity across several datasets. These findings underscore the potential of the Rotated Vector Quantization framework to advance image analysis applications and broaden the horizons for future research in the field of representation learning.

\end{abstract}